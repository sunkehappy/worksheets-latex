% templates/ws_icons.tex
% SVG-based icons for worksheets (using PDF conversion for compatibility)
\makeatletter
\@ifundefined{wsiconsloaded}{%

% 使用 graphicx 包来插入图片(PDF 格式)
% SVG 文件需要预先转换为 PDF(可以使用脚本自动转换)
\usepackage{graphicx}

% ===== SVG 图标插入宏 =====
% \WSDrawIcon{name}{scale}
% 从 icons/ 目录加载 PDF 文件(由 SVG 转换而来)
% 优先使用 PDF,如果不存在则尝试 SVG(如果系统支持)
\def\WSDrawIcon#1#2{%
  % 优先使用 PDF 版本(推荐:预先转换 SVG 为 PDF)
  \IfFileExists{../icons/#1.pdf}{%
    \resizebox{#2em}{#2em}{%
      \includegraphics{../icons/#1.pdf}%
    }%
  }{%
    % 如果 PDF 不存在,尝试 SVG(需要系统支持)
    \IfFileExists{../icons/#1.svg}{%
      % 尝试使用 svg 包(如果可用)
      \@ifpackageloaded{svg}{%
        \resizebox{#2em}{#2em}{%
          \includesvg[inkscapeformat=pdf,inkscapelatex=false]{../icons/#1}%
        }%
      }{%
        % 如果 svg 包不可用,显示占位符并提示
        \fbox{\rule{#2em}{#2em}}%
        \PackageWarning{wsicons}{Icon #1.svg found but svg package not available. Please convert to PDF.}%
      }%
    }{%
      % 如果文件都不存在,显示占位符
      \fbox{\rule{#2em}{#2em}}%
      \PackageWarning{wsicons}{Icon #1 not found (tried .pdf and .svg)}%
    }%
  }%
}

% ===== N 个图标排成一行(自动留间距) =====
% \WSIconRow{name}{count}{scale em}{gap em}
\def\WSIconRow#1#2#3#4{%
  \foreach \i in {1,...,#2}{\WSDrawIcon{#1}{#3}\hspace{#4 em}}%
}

% ===== 题目:图形加法 =====
% \WSPictureAddIcons{nameL}{nL}{nameR}{nR}
\def\WSPictureAddIcons#1#2#3#4{%
  \WSIconRow{#1}{#2}{1.6}{0.4}%
  \hspace{0.6em}{\Large +}\hspace{0.6em}%
  \WSIconRow{#3}{#4}{1.6}{0.4}%
  \hspace{0.6em}{\Large =}\hspace{0.6em}%
  \fbox{\rule{0pt}{10pt}\rule{42pt}{0pt}}%
}

% ===== 若数量较大时,自动换两行 =====
% \WSPictureAddIconsWrap{name}{count}{perRow}
\def\WSPictureAddIconsWrap#1#2#3{%
  \def\c{#2}%
  \def\per{#3}%
  % 计算行数与余数为整数,便于 \ifnum 判断
  \pgfmathtruncatemacro{\rowsint}{ceil(\c/\per)}%
  \pgfmathtruncatemacro{\lastint}{mod(\c,\per)}%
  \ifnum\rowsint=1
    \WSIconRow{#1}{#2}{1.6}{0.4}%
  \else
    % 第一行
    \WSIconRow{#1}{#3}{1.6}{0.4}%
    % 第二行(余数)- 使用 \linebreak[0] 强制换行,在表格单元格中安全
    \ifnum\lastint=0
      \linebreak[0]\vskip-2pt\WSIconRow{#1}{#3}{1.6}{0.4}%
    \else
      \linebreak[0]\vskip-2pt\WSIconRow{#1}{\lastint}{1.6}{0.4}%
    \fi
  \fi
}

% ===== 换行版的题目 =====
% \WSPictureAddIcons2L{nameL}{nL}{nameR}{nR}{perRow}
\def\WSPictureAddIcons2L#1#2#3#4#5{%
  \WSPictureAddIconsWrap{#1}{#2}{#5}%
  \hspace{0.3em}{\Large +}\hspace{0.3em}%
  \WSPictureAddIconsWrap{#3}{#4}{#5}%
  \hspace{0.3em}{\Large =}\hspace{0.3em}%
  \fbox{\rule{0pt}{10pt}\rule{32pt}{0pt}}%
}

% ===== 一行一个算式:左边图形,中间加号,右边空白算式 =====
% \WSPictureAddEquation{nameL}{nL}{nameR}{nR}{perRow}
\def\WSPictureAddEquation#1#2#3#4#5{%
  % 使用表格布局:左图形 | 加号 | 右图形 | 空白算式
  % 所有列都使用 m 列类型(垂直居中),确保加号和等式与最高的图标列垂直居中
  % 加号左边间距较小,右边较大,使加号偏左,右侧图形偏右
  % 使用 \arrayrulewidth=0pt 来隐藏表格的所有横线和竖线
  \begingroup
  \setlength{\arrayrulewidth}{0pt}%
  \renewcommand{\arraystretch}{1}%
  \begin{tabular*}{\linewidth}{@{}>{\raggedright\arraybackslash}m{0.27\linewidth}@{\hspace{0em}}>{\centering\arraybackslash}m{0.08\linewidth}@{\hspace{0em}}>{\raggedright\arraybackslash}m{0.27\linewidth}@{\extracolsep{\fill}}>{\raggedleft\arraybackslash}m{0.30\linewidth}@{}}
    \WSPictureAddIconsWrapLarge{#1}{#2}{#5}%
    &
    {\LARGE\, + \,}%
    &
    \WSPictureAddIconsWrapLarge{#3}{#4}{#5}%
    &
    \raisebox{-3.5pt}{\rule{30pt}{1pt}}\hspace{0.1em}{\LARGE\, + \,}\hspace{0.1em}%
    \raisebox{-3.5pt}{\rule{30pt}{1pt}}\hspace{0.1em}{\LARGE\, = \,}\hspace{0.1em}%
    \raisebox{-3.5pt}{\rule{30pt}{1pt}}%
  \end{tabular*}%
  \endgroup
}

% ===== 图形换行版本(用于算式,换行后左对齐,尺寸调小以便对齐) =====
\def\WSPictureAddIconsWrapLarge#1#2#3{%
  \def\c{#2}%
  \def\per{#3}%
  \pgfmathtruncatemacro{\rowsint}{ceil(\c/\per)}%
  \pgfmathtruncatemacro{\lastint}{mod(\c,\per)}%
  \ifnum\rowsint=1
    \WSIconRow{#1}{#2}{2.3}{0.2}%
  \else
    % 第一行
    \WSIconRow{#1}{#3}{2.3}{0.2}%
    % 第二行(余数)- 换行后左对齐,便于数数
    \ifnum\lastint=0
      \par\vskip-2pt\noindent\WSIconRow{#1}{#3}{2.3}{0.2}%
    \else
      \par\vskip-2pt\noindent\WSIconRow{#1}{\lastint}{2.3}{0.2}%
    \fi
  \fi
}

% 需要 tikz 包用于 \foreach 和数学计算
\@ifpackageloaded{tikz}{}{\usepackage{tikz}}
\@ifpackageloaded{xfp}{}{\usepackage{xfp}}

% Backward-compatible aliases (define only if currently undefined)
\@ifundefined{PictureAddIcons}{\def\PictureAddIcons#1#2#3#4{\WSPictureAddIcons{#1}{#2}{#3}{#4}}}{}
\@ifundefined{PictureAddIcons2L}{\def\PictureAddIcons2L#1#2#3#4#5{\WSPictureAddIcons2L{#1}{#2}{#3}{#4}{#5}}}{}
\@ifundefined{IconRow}{\def\IconRow#1#2#3#4{\WSIconRow{#1}{#2}{#3}{#4}}}{}
\@ifundefined{drawIcon}{\def\drawIcon#1#2{\WSDrawIcon{#1}{#2}}}{}

\def\wsiconsloaded{}
}{}
\makeatother
