% templates/icons.tex
\makeatletter
\@ifundefined{wsiconsloaded}{%
\@ifpackageloaded{tikz}{}{\usepackage{tikz}}

% 通用:单个图标的线宽(黑白打印友好)
\def\wsiconlinewidth{0.9pt}

% ===== 基础形状 =====
% Apple(圆+叶+茎)
\def\wsiconApple#1{% #1 = scale
  \begin{tikzpicture}[scale=#1, baseline={(0,-0.1)}]
    \draw[line width=\wsiconlinewidth] (0,0) circle (0.5);
    \draw[line width=\wsiconlinewidth] (0,0.5) -- (0,0.8);
    \draw[line width=\wsiconlinewidth] (0.15,0.85) .. controls (0.5,1.05) .. (0.35,0.68) -- cycle;
    \draw[line width=\wsiconlinewidth] (0.15,0.85) .. controls (0.5,1.05) .. (0.35,0.68) -- cycle;
  \end{tikzpicture}%
}

% Star(五角星)
\def\wsiconStar#1{% #1 = scale
  \begin{tikzpicture}[scale=#1, baseline={(0,-0.1)}]
    \draw[line width=\wsiconlinewidth]
      (90:0.55) -- (162:0.55) -- (234:0.55) -- (306:0.55) -- (18:0.55) -- cycle;
  \end{tikzpicture}%
}

% Triangle(等边三角形)
\def\wsiconTriangle#1{% #1 = scale
  \begin{tikzpicture}[scale=#1, baseline={(0,-0.1)}]
    \draw[line width=\wsiconlinewidth] (0,0.58) -- (-0.5,-0.29) -- (0.5,-0.29) -- cycle;
  \end{tikzpicture}%
}

% ===== 调度器:按名称选择图标 =====
% #1 = name, #2 = scale
\def\WSDrawIcon#1#2{%
  \expandafter\ifx\csname wsicon#1\endcsname\relax
    \wsiconApple{#2}%
  \else
    \csname wsicon#1\endcsname{#2}%
  \fi
}

% ===== N 个图标排成一行(自动留间距) =====
% \WSIconRow{name}{count}{scale}{gap em}
\def\WSIconRow#1#2#3#4{%
  \foreach \i in {1,...,#2}{\WSDrawIcon{#1}{#3}\hspace{#4 em}}%
}

% ===== 题目:图形加法 =====
% \WSPictureAddIcons{nameL}{nL}{nameR}{nR}
\def\WSPictureAddIcons#1#2#3#4{%
  \WSIconRow{#1}{#2}{0.75}{0.4}%
  \hspace{0.6em}{\Large +}\hspace{0.6em}%
  \WSIconRow{#3}{#4}{0.75}{0.4}%
  \hspace{0.6em}{\Large =}\hspace{0.6em}%
  \fbox{\rule{0pt}{10pt}\rule{42pt}{0pt}}%
}

% ===== 若数量较大时,自动换两行 =====
% \WSPictureAddIconsWrap{name}{count}{perRow}
\def\WSPictureAddIconsWrap#1#2#3{%
  \def\c{#2}%
  \def\per{#3}%
  % 计算行数与余数为整数,便于 \ifnum 判断
  \pgfmathtruncatemacro{\rowsint}{ceil(\c/\per)}%
  \pgfmathtruncatemacro{\lastint}{mod(\c,\per)}%
  \ifnum\rowsint=1
    \WSIconRow{#1}{#2}{0.75}{0.4}%
  \else
    % 第一行
    \WSIconRow{#1}{#3}{0.75}{0.4}\\[-2pt]
    % 第二行(余数)
    \ifnum\lastint=0
      \WSIconRow{#1}{#3}{0.75}{0.4}%
    \else
      \WSIconRow{#1}{\lastint}{0.75}{0.4}%
    \fi
  \fi
}

% ===== 换行版的题目 =====
% \WSPictureAddIcons2L{nameL}{nL}{nameR}{nR}{perRow}
\def\WSPictureAddIcons2L#1#2#3#4#5{%
  \WSPictureAddIconsWrap{#1}{#2}{#5}%
  \hspace{0.6em}{\Large +}\hspace{0.6em}%
  \WSPictureAddIconsWrap{#3}{#4}{#5}%
  \hspace{0.6em}{\Large =}\hspace{0.6em}%
  \fbox{\rule{0pt}{10pt}\rule{42pt}{0pt}}%
}

% Backward-compatible aliases (define only if currently undefined)
\@ifundefined{PictureAddIcons}{\def\PictureAddIcons#1#2#3#4{\WSPictureAddIcons{#1}{#2}{#3}{#4}}}{}
\@ifundefined{PictureAddIcons2L}{\def\PictureAddIcons2L#1#2#3#4#5{\WSPictureAddIcons2L{#1}{#2}{#3}{#4}{#5}}}{}
\@ifundefined{IconRow}{\def\IconRow#1#2#3#4{\WSIconRow{#1}{#2}{#3}{#4}}}{}
\@ifundefined{drawIcon}{\def\drawIcon#1#2{\WSDrawIcon{#1}{#2}}}{}

\def\wsiconsloaded{}
}{}
\makeatother
