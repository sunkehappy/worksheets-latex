% templates/ws_icons.tex
% SVG-based icons for worksheets (using PDF conversion for compatibility)
\makeatletter
\@ifundefined{wsiconsloaded}{%

% 使用 graphicx 包来插入图片(PDF 格式)
% SVG 文件需要预先转换为 PDF(可以使用脚本自动转换)
\usepackage{graphicx}

% ===== SVG 图标插入宏 =====
% \WSDrawIcon{name}{scale}
% 从 icons/ 目录加载 PDF 文件(由 SVG 转换而来)
% 优先使用 PDF,如果不存在则尝试 SVG(如果系统支持)
\def\WSDrawIcon#1#2{%
  % 优先使用 PDF 版本(推荐:预先转换 SVG 为 PDF)
  \IfFileExists{../icons/#1.pdf}{%
    \resizebox{#2em}{#2em}{%
      \includegraphics{../icons/#1.pdf}%
    }%
  }{%
    % 如果 PDF 不存在,尝试 SVG(需要系统支持)
    \IfFileExists{../icons/#1.svg}{%
      % 尝试使用 svg 包(如果可用)
      \@ifpackageloaded{svg}{%
        \resizebox{#2em}{#2em}{%
          \includesvg[inkscapeformat=pdf,inkscapelatex=false]{../icons/#1}%
        }%
      }{%
        % 如果 svg 包不可用,显示占位符并提示
        \fbox{\rule{#2em}{#2em}}%
        \PackageWarning{wsicons}{Icon #1.svg found but svg package not available. Please convert to PDF.}%
      }%
    }{%
      % 如果文件都不存在,显示占位符
      \fbox{\rule{#2em}{#2em}}%
      \PackageWarning{wsicons}{Icon #1 not found (tried .pdf and .svg)}%
    }%
  }%
}

% ===== N 个图标排成一行(自动留间距) =====
% \WSIconRow{name}{count}{scale em}{gap em}
\def\WSIconRow#1#2#3#4{%
  \foreach \i in {1,...,#2}{\WSDrawIcon{#1}{#3}\hspace{#4 em}}%
}

% ===== 题目:图形加法 =====
% \WSPictureAddIcons{nameL}{nL}{nameR}{nR}
\def\WSPictureAddIcons#1#2#3#4{%
  \WSIconRow{#1}{#2}{1.6}{0.4}%
  \hspace{0.6em}{\Large +}\hspace{0.6em}%
  \WSIconRow{#3}{#4}{1.6}{0.4}%
  \hspace{0.6em}{\Large =}\hspace{0.6em}%
  \fbox{\rule{0pt}{10pt}\rule{42pt}{0pt}}%
}

% ===== 若数量较大时,自动换两行 =====
% \WSPictureAddIconsWrap{name}{count}{perRow}
\def\WSPictureAddIconsWrap#1#2#3{%
  \def\c{#2}%
  \def\per{#3}%
  % 计算行数与余数为整数,便于 \ifnum 判断
  \pgfmathtruncatemacro{\rowsint}{ceil(\c/\per)}%
  \pgfmathtruncatemacro{\lastint}{mod(\c,\per)}%
  \ifnum\rowsint=1
    \WSIconRow{#1}{#2}{1.6}{0.4}%
  \else
    % 第一行
    \WSIconRow{#1}{#3}{1.6}{0.4}%
    % 第二行(余数)- 使用 \linebreak[0] 强制换行,在表格单元格中安全
    \ifnum\lastint=0
      \linebreak[0]\vskip-2pt\WSIconRow{#1}{#3}{1.6}{0.4}%
    \else
      \linebreak[0]\vskip-2pt\WSIconRow{#1}{\lastint}{1.6}{0.4}%
    \fi
  \fi
}

% ===== 换行版的题目 =====
% \WSPictureAddIcons2L{nameL}{nL}{nameR}{nR}{perRow}
\def\WSPictureAddIcons2L#1#2#3#4#5{%
  \WSPictureAddIconsWrap{#1}{#2}{#5}%
  \hspace{0.3em}{\Large +}\hspace{0.3em}%
  \WSPictureAddIconsWrap{#3}{#4}{#5}%
  \hspace{0.3em}{\Large =}\hspace{0.3em}%
  \fbox{\rule{0pt}{10pt}\rule{32pt}{0pt}}%
}

% ===== 一行一个算式:左边图形,中间加号,右边空白算式 =====
% \WSPictureAddEquation{nameL}{nL}{nameR}{nR}{perRow}
\def\WSPictureAddEquation#1#2#3#4#5{%
  % 使用表格布局:左图形 | 加号 | 右图形 | 空白算式
  % 所有列都使用 m 列类型(垂直居中),确保加号和等式与最高的图标列垂直居中
  % 加号左边间距较小,右边较大,使加号偏左,右侧图形偏右
  % 使用 \arrayrulewidth=0pt 来隐藏表格的所有横线和竖线
  \begingroup
  \setlength{\arrayrulewidth}{0pt}%
  \renewcommand{\arraystretch}{1}%
  \begin{tabular*}{\linewidth}{@{}>{\raggedright\arraybackslash}m{0.27\linewidth}@{\hspace{0em}}>{\centering\arraybackslash}m{0.08\linewidth}@{\hspace{0em}}>{\raggedright\arraybackslash}m{0.27\linewidth}@{\extracolsep{\fill}}>{\raggedleft\arraybackslash}m{0.30\linewidth}@{}}
    \WSPictureAddIconsWrapLarge{#1}{#2}{#5}%
    &
    {\LARGE\, + \,}%
    &
    \WSPictureAddIconsWrapLarge{#3}{#4}{#5}%
    &
    \raisebox{-3.5pt}{\rule{30pt}{1pt}}\hspace{0.1em}{\LARGE\, + \,}\hspace{0.1em}%
    \raisebox{-3.5pt}{\rule{30pt}{1pt}}\hspace{0.1em}{\LARGE\, = \,}\hspace{0.1em}%
    \raisebox{-3.5pt}{\rule{30pt}{1pt}}%
  \end{tabular*}%
  \endgroup
}

% ===== 答案版:左边图形,右边直接显示数字算式 =====
% \WSPictureAddEquationSolved{nameL}{nL}{nameR}{nR}{perRow}
\def\WSPictureAddEquationSolved#1#2#3#4#5{%
  \begingroup
  \setlength{\arrayrulewidth}{0pt}%
  \renewcommand{\arraystretch}{1}%
  \begin{tabular*}{\linewidth}{@{}>{\raggedright\arraybackslash}m{0.27\linewidth}@{\hspace{0em}}>{\centering\arraybackslash}m{0.08\linewidth}@{\hspace{0em}}>{\raggedright\arraybackslash}m{0.27\linewidth}@{\extracolsep{\fill}}>{\raggedleft\arraybackslash}m{0.30\linewidth}@{}}
    \WSPictureAddIconsWrapLarge{#1}{#2}{#5}%
    &
    {\LARGE\, + \,}%
    &
    \WSPictureAddIconsWrapLarge{#3}{#4}{#5}%
    &
    {\LARGE\ #2\ \,+\ \,#4\ \;=\ \;\fpeval{#2 + #4}}%
  \end{tabular*}%
  \endgroup
}

% ===== 图形换行版本(用于算式,换行后左对齐,尺寸调小以便对齐) =====
\def\WSPictureAddIconsWrapLarge#1#2#3{%
  \def\c{#2}%
  \def\per{#3}%
  \pgfmathtruncatemacro{\rowsint}{ceil(\c/\per)}%
  \pgfmathtruncatemacro{\lastint}{mod(\c,\per)}%
  \ifnum\rowsint=1
    \WSIconRow{#1}{#2}{2.3}{0.2}%
  \else
    % 第一行
    \WSIconRow{#1}{#3}{2.3}{0.2}%
    % 第二行(余数)- 换行后左对齐,便于数数
    \ifnum\lastint=0
      \par\vskip-2pt\noindent\WSIconRow{#1}{#3}{2.3}{0.2}%
    \else
      \par\vskip-2pt\noindent\WSIconRow{#1}{\lastint}{2.3}{0.2}%
    \fi
  \fi
}

% 需要 tikz 包用于 \foreach 和数学计算
\@ifpackageloaded{tikz}{}{\usepackage{tikz}}
\@ifpackageloaded{xfp}{}{\usepackage{xfp}}

% ===== 数字线相关宏 =====
% 基础数字线(0-maxValue,带刻度)
% 根据maxValue自动调整缩放比例
% \WSNumberLine{maxValue}
\def\WSNumberLine#1{%
  \ifnum#1>15
    \def\scalevalue{0.85}%
    \pgfmathsetmacro{\unitwidth}{0.7}%
  \else
    \ifnum#1>10
      \def\scalevalue{0.95}%
      \pgfmathsetmacro{\unitwidth}{0.75}%
    \else
      \def\scalevalue{1.0}%
      \pgfmathsetmacro{\unitwidth}{0.7}%
    \fi
  \fi
  \begin{tikzpicture}[scale=\scalevalue, baseline=-0.5ex]
    % 绘制水平线
    \pgfmathsetmacro{\maxx}{#1*\unitwidth+0.5*\unitwidth}%
    \pgfmathsetmacro{\arrowx}{#1*\unitwidth+\unitwidth}%
    \draw[->] (0,0) -- (\maxx,0);
    \draw[<-] (\maxx,0) -- (\arrowx,0);
    % 绘制刻度线和数字
    \foreach \x in {0,1,...,#1} {
      \pgfmathsetmacro{\xpos}{\x*\unitwidth}%
      \draw (\xpos,0.1) -- (\xpos,-0.1);
      \node[below] at (\xpos,-0.2) {\small \x};
    }
  \end{tikzpicture}%
}

% 带弧线的数字线(示例:显示从0到first,然后从first到result)
% \WSNumberLineWithArcs{first}{second}{result}{maxValue}
\def\WSNumberLineWithArcs#1#2#3#4{%
  \ifnum#4>15
    \def\scalevalue{0.8}%
    \pgfmathsetmacro{\unitwidth}{0.65}%
  \else
    \ifnum#4>10
      \def\scalevalue{0.9}%
      \pgfmathsetmacro{\unitwidth}{0.7}%
    \else
      \def\scalevalue{1.0}%
      \pgfmathsetmacro{\unitwidth}{0.7}%
    \fi
  \fi
  % 预先计算控制点坐标
  \pgfmathsetmacro{\firstx}{#1*\unitwidth}%
  \pgfmathsetmacro{\resultx}{#3*\unitwidth}%
  \pgfmathsetmacro{\firstctrlone}{\firstx*0.3}%
  \pgfmathsetmacro{\firstctrltwo}{\firstx*0.7}%
  \pgfmathsetmacro{\secondctrlone}{\firstx+(\resultx-\firstx)*0.3}%
  \pgfmathsetmacro{\secondctrltwo}{\firstx+(\resultx-\firstx)*0.7}%
  \begin{tikzpicture}[scale=\scalevalue, baseline=-0.5ex]
    % 绘制水平线
    \pgfmathsetmacro{\maxx}{#4*\unitwidth+0.5*\unitwidth}%
    \pgfmathsetmacro{\arrowx}{#4*\unitwidth+\unitwidth}%
    \draw[->] (0,0) -- (\maxx,0);
    \draw[<-] (\maxx,0) -- (\arrowx,0);
    % 绘制刻度线和数字
    \foreach \x in {0,1,...,#4} {
      \pgfmathsetmacro{\xpos}{\x*\unitwidth}%
      \draw (\xpos,0.1) -- (\xpos,-0.1);
      \node[below] at (\xpos,-0.2) {\small \x};
    }
    % 绘制第一个弧线(从0到first)
    \draw[thick,->,red] (0,0) .. controls (\firstctrlone,0.8) and (\firstctrltwo,0.8) .. (\firstx,0);
    % 绘制第二个弧线(从first到result)
    \draw[thick,->,red] (\firstx,0) .. controls (\secondctrlone,0.6) and (\secondctrltwo,0.6) .. (\resultx,0);
  \end{tikzpicture}%
}

% 空白数字线(练习题:画红色弧线表示几加几)
% \WSNumberLineBlank{first}{second}{maxValue}
\def\WSNumberLineBlank#1#2#3{%
  \ifnum#3>15
    \def\scalevalue{0.8}%
    \pgfmathsetmacro{\unitwidth}{0.65}%
  \else
    \ifnum#3>10
      \def\scalevalue{0.9}%
      \pgfmathsetmacro{\unitwidth}{0.7}%
    \else
      \def\scalevalue{1.0}%
      \pgfmathsetmacro{\unitwidth}{0.7}%
    \fi
  \fi
  % 计算结果
  \pgfmathsetmacro{\result}{#1+#2}%
  % 预先计算控制点坐标
  \pgfmathsetmacro{\firstx}{#1*\unitwidth}%
  \pgfmathsetmacro{\resultx}{\result*\unitwidth}%
  \pgfmathsetmacro{\firstctrlone}{\firstx*0.3}%
  \pgfmathsetmacro{\firstctrltwo}{\firstx*0.7}%
  \pgfmathsetmacro{\secondctrlone}{\firstx+(\resultx-\firstx)*0.3}%
  \pgfmathsetmacro{\secondctrltwo}{\firstx+(\resultx-\firstx)*0.7}%
  \begin{tikzpicture}[scale=\scalevalue, baseline=-0.5ex]
    % 绘制水平线
    \pgfmathsetmacro{\maxx}{#3*\unitwidth+0.5*\unitwidth}%
    \pgfmathsetmacro{\arrowx}{#3*\unitwidth+\unitwidth}%
    \draw[->] (0,0) -- (\maxx,0);
    \draw[<-] (\maxx,0) -- (\arrowx,0);
    % 绘制刻度线和数字
    \foreach \x in {0,1,...,#3} {
      \pgfmathsetmacro{\xpos}{\x*\unitwidth}%
      \draw (\xpos,0.1) -- (\xpos,-0.1);
      \node[below] at (\xpos,-0.2) {\small \x};
    }
    % 绘制第一个弧线(从0到first)
    \draw[thick,->,red] (0,0) .. controls (\firstctrlone,0.8) and (\firstctrltwo,0.8) .. (\firstx,0);
    % 绘制第二个弧线(从first到result)
    \draw[thick,->,red] (\firstx,0) .. controls (\secondctrlone,0.6) and (\secondctrltwo,0.6) .. (\resultx,0);
  \end{tikzpicture}%
}

% 数字线加法题目(一行一个:左边数字线,右边空白算式)
% \WSNumberLineAddEquation{first}{second}{maxValue}
\def\WSNumberLineAddEquation#1#2#3{%
  \begingroup
  \setlength{\arrayrulewidth}{0pt}%
  \renewcommand{\arraystretch}{1}%
  \begin{tabular*}{\linewidth}{@{}>{\centering\arraybackslash}m{0.58\linewidth}@{\hspace{0.02\linewidth}}>{\raggedleft\arraybackslash}m{0.38\linewidth}@{}}
    \WSNumberLineBlank{#1}{#2}{#3}%
    &
    \raisebox{-3.5pt}{\rule{30pt}{1pt}}\hspace{0.1em}{\LARGE\, + \,}\hspace{0.1em}%
    \raisebox{-3.5pt}{\rule{30pt}{1pt}}\hspace{0.1em}{\LARGE\, = \,}\hspace{0.1em}%
    \raisebox{-3.5pt}{\rule{30pt}{1pt}}%
  \end{tabular*}%
  \endgroup
}

% 数字+下划线的组合(用于样例)
% \WSNumberWithUnderline{number}
\def\WSNumberWithUnderline#1{%
  \raisebox{-3.5pt}{%
    \hbox to 30pt{%
      \hfil
      \vbox{%
        \hbox{\LARGE\ #1}%
        \hbox{\rule{30pt}{1pt}}%
      }%
      \hfil
    }%
  }%
}

% 数字线加法示例(带弧线)
% \WSNumberLineAddExample{first}{second}{result}{maxValue}
\def\WSNumberLineAddExample#1#2#3#4{%
  \begingroup
  \setlength{\arrayrulewidth}{0pt}%
  \renewcommand{\arraystretch}{1}%
  \begin{tabular*}{\linewidth}{@{}>{\centering\arraybackslash}m{0.58\linewidth}@{\hspace{0.02\linewidth}}>{\raggedleft\arraybackslash}m{0.38\linewidth}@{}}
    \WSNumberLineWithArcs{#1}{#2}{#3}{#4}%
    &
    \WSNumberWithUnderline{#1}\hspace{0.1em}{\LARGE\, + \,}\hspace{0.1em}%
    \WSNumberWithUnderline{#2}\hspace{0.1em}{\LARGE\, = \,}\hspace{0.1em}%
    \WSNumberWithUnderline{#3}%
  \end{tabular*}%
  \endgroup
}

% 数字线加法题目(空数字线,右边显示计算式让学生画线)
% \WSNumberLineAddDraw{first}{second}{maxValue}
\def\WSNumberLineAddDraw#1#2#3{%
  \begingroup
  \setlength{\arrayrulewidth}{0pt}%
  \renewcommand{\arraystretch}{1}%
  \begin{tabular*}{\linewidth}{@{}>{\centering\arraybackslash}m{0.58\linewidth}@{\hspace{0.02\linewidth}}>{\raggedleft\arraybackslash}m{0.38\linewidth}@{}}
    \WSNumberLine{#3}%
    &
    {\LARGE\ #1\ \,+\ \,#2\ \;=\ \;}\raisebox{-3.5pt}{\rule{30pt}{1pt}}%
  \end{tabular*}%
  \endgroup
}

% Backward-compatible aliases (define only if currently undefined)
\@ifundefined{PictureAddIcons}{\def\PictureAddIcons#1#2#3#4{\WSPictureAddIcons{#1}{#2}{#3}{#4}}}{}
\@ifundefined{PictureAddIcons2L}{\def\PictureAddIcons2L#1#2#3#4#5{\WSPictureAddIcons2L{#1}{#2}{#3}{#4}{#5}}}{}
\@ifundefined{IconRow}{\def\IconRow#1#2#3#4{\WSIconRow{#1}{#2}{#3}{#4}}}{}
\@ifundefined{drawIcon}{\def\drawIcon#1#2{\WSDrawIcon{#1}{#2}}}{}

\def\wsiconsloaded{}
}{}
\makeatother
